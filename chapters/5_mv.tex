\let\textcircled=\pgftextcircled
\chapter{Multi-variable Emulation}
\label{chap:mv}

\section{Introduction}

ML has been shown to be effective in emulating high-resolution climate model even for a highly variable and stochastic setting like UK precipitation.
However, it is often important to have coherent samples of more than one variable. For example, heat stress may involve air temp, humidity and wind speed \cite{Blazejczyk2012UTCIcomparison}. Indices have been created to simplify complex joint distributions into a single value. This allows a method for evaluating joint samples from a multivariate emulator - as well as modelling the marginal distributions of each variable and capturing some elements of the joint distribution, ideally the emulator should be able to recreate the same distribution of a compound index.
These indicies also allow us to focus on high impact events - for example it is more important to capture the distribution of stressfully hot days rather than small variations on more temperate days.

\section{Method}

\subsection{Data}

Variables

\begin{itemize}
    \item Temp at 1.5m
    \item relhum at 1.5m
    \item precip
\end{itemize}

Indices

\begin{itemize}
    \item SWBGT
\end{itemize}